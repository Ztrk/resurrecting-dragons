\documentclass[12pt]{article}
\usepackage[utf8]{inputenc}
\usepackage[T1]{fontenc}
\usepackage{lmodern}
\usepackage{amsmath}

\usepackage[svgnames]{xcolor}
\usepackage[a4paper,bindingoffset=0.2in,%
left=1in,right=1in,top=1in,bottom=1.5in,%
footskip=.25in]{geometry}
\pagenumbering{gobble}
\usepackage[colorlinks=true, linkcolor=Black, urlcolor=Blue]{hyperref}

\newcommand{\state}[1]{\texttt{\textbf{#1}}}
\newcommand{\msg}[1]{\texttt{\emph{#1}}}
\newcommand{\var}[1]{\texttt{#1}}

\begin{document}
\title{Algorytm - Wskrzeszanie smoków}
\author{Sebastian Michoń 136770, Marcin Zatorski 136834}
\date{}
\maketitle

\section{Wstęp}
\subsection{Założenia}
	\begin{enumerate}
		\item Biura o liczności $b$ i szkielety o liczności $s$ są zasobami nierozróżnialnymi.
		\item Procesy wiedzą jaką rolę pełni każdy inny proces.
		\item Proces o specjalizacji x będzie nazywany "proces-x" (np. proces o specjalizacji głowa to proces-głowa)
		\item Zlecenia są rozróżnialne i mają przypisane id (numerowane od 1), zgodnie z kolejnością generacji.
		\item Liczby procesów o poszczególnych specjalizacjach będą oznaczane przez: procesy-głowy: $n_g$, procesy-tułowie: $n_t$, procesy-ogony: $n_o$.
	\end{enumerate}

\section{Zmienne}
\begin{description}
	\item[$z_p$] Liczba otrzymanych wiadomości \msg{ZLECENIE}
	\item[$z_c$] Liczba otrzymanych wiadomości \msg{REQ\_ZLECENIE} o priorytecie większym
	\item[$cnt$] Liczba otrzymanych wiadomości \msg{REQ\_ZLECENIE} o priorytecie mniejszym
	
	\item[\var{QUEUE\_ZLECENIE}, \var{QUEUE\_BIURO}, \var{QUEUE\_SZKIELET}] Kolejki procesów oczekujących na wiadomość \msg{ACK}
	\item[\var{QUEUE\_OFFER}] Kolejka procesów od których otrzymaliśmy wiadomość \msg{OFFER}.
\end{description}

\section{Wiadomości}
\begin{description}
	\item[\msg{ZLECENIE}] Informacja o nowym zleceniu
	\item[\msg{REQ\_ZLECENIE}] Informacja o chęci dostępu do zlecenia
	\item[\msg{ACK\_ZLECENIE}] Zaakceptowanie dostępu do zlecenia
	\item[\msg{OFFER}] Oferta przyłączenia do grupy, zawiera id zlecenia
	\item[\msg{REQ\_BIURO}] Prośba o dostęp do biura
	\item[\msg{ACK\_BIURO}] Zaakceptowanie dostępu do biura
	\item[\msg{KONIEC\_BIURO}] Informacja o końcu pracy w biurze
	\item[\msg{REQ\_SZKIELET}] Prośba o dostęp do szkieletu
	\item[\msg{ACK\_SZKIELET}] Zaakceptowanie dostępu do szkieletu
	\item[\msg{START\_SZKIELET}] Rozpoczęcie pracy ze szkieletem
	\item[\msg{KONIEC\_SZKIELET}] Koniec pracy ze szkieletem
\end{description}
Wiadomości \msg{REQ\_ZLECENIE}, \msg{REQ\_BIURO}, \msg{REQ\_SZKIELET} zawierają priorytet procesu - parę: zegar Lamporta, identyfikator procesu. Wiadomości \msg{ACK\_ZLECENIE}, \msg{ACK\_BIURO}, \msg{ACK\_SZKIELET} zawierają informację o wiadomości \msg{REQ}, na którą są odpowiedzią.

\section{Stany}
\begin{description}
	\item[\state{START}] Stan początkowy procesu, zanim zacznie ubieganie się o zlecenie.
	\item[\state{CZEKAJ\_ZLECENIE}] Stan oczekiwania na sekcję krytyczną zleceń.
	\item[\state{MAM\_ZLECENIE}] Stan po otrzymaniu zlecenia, przed przydzieleniem do grupy.
	\item[\state{CZEKAJ}] Stan po dobraniu do grupy, czekanie na skończenie pracy w biurze i rozpoczęcie pracy ze szkieletem
	\item[\state{CZEKAJ\_BIURO}] Oczekiwanie na otrzymanie dostępu do biura
	\item[\state{PRACA\_BIURO}] Praca w biurze
	\item[\state{CZEKAJ\_SZKIELET}] Oczekiwanie na otrzymanie dostępu do szkieletu
	\item[\state{PRACA\_SZKIELET}] Praca ze szkieletem i wskrzeszanie smoka
\end{description}

\section{Wskrzeszanie}

\subsection{Dostęp do zleceń i dobór do grupy}
\begin{enumerate}
	\item Generator co jakiś czas wysyła wiadomość \msg{ZLECENIE} do wszystkich procesów poza samym sobą.
	
	\item Pozostałe procesy zaczynają w stanie \state{START}.
	
	Reakcje na wiadomości w stanie \state{START}:
	\begin{enumerate}
		\item \msg{REQ\_SZKIELET} Odpowiedz komunikatem \msg{ACK\_SZKIELET}
		\item \msg{REQ\_BIURO} Odpowiedz komunikatem \msg{ACK\_BIURO}
		\item \msg{REQ\_ZLECENIE} Odpowiedz komunikatem \msg{ACK\_ZLECENIE}. Zwiększ wartość $z_c$ o 1.
		\item \msg{OFFER} Dodaj id procesu do lokalnej mapy ofert w pozycji będącej id taska, którego dotyczy oferta.
		\item \msg{ZLECENIE} Zwiększ wartość $z_p$ o 1
		\item \msg{ACK\_BIURO}, \msg{ACK\_SZKIELET}, \msg{ACK\_ZLECENIE} Zignoruj
	\end{enumerate}
		
	\item Proces w stanie \state{START} wysyła wiadomość \msg{REQ\_ZLECENIE} do wszystkich procesów o tej samej specjalizacji i przechodzi do stanu \state{CZEKAJ\_ZLECENIE}.
	
	Reakcje na wiadomości w stanie \state{CZEKAJ\_ZLECENIE}:
	\begin{enumerate}
		\item \msg{REQ\_ZLECENIE} Wyślij \msg{ACK\_ZLECENIE}. Jeśli odebrana prośba ma priorytet mniejszy to zwiększ $cnt$ o 1. W przeciwnym wypadku zwiększ $z_c$ o 1.
		\item \msg{ACK\_ZLECENIE} Jeśli jest to odpowiedź na ostatnią wysłaną wiadomość \msg{REQ\_ZLECENIE} zwiększ licznik ACK o 1, w p.p. zignoruj.
		\item Pozostałe jak w stanie \state{START}.
	\end{enumerate}
		
	\item Po otrzymaniu $n_x - 1$ odpowiedzi \msg{ACK\_ZLECENIE} proces przechodzi do stanu \state{MAM\_ZLECENIE}. Id zlecenia jest równe $z_c + 1$ (to zlecenie może jeszcze nie być wygenerowane). Następnie proces aktualizuje zmienne: $z_c$ += $cnt + 1$, $cnt = 0$.
	\begin{description}
		\item[$n_x$] liczba procesów o tej samej specjalizacji
		\item[$z_c$] liczba otrzymanych wiadomości \msg{REQ\_ZLECENIE} o priorytecie większym.
	\end{description}
			
	\item Reakcje na wiadomości w stanie \state{MAM\_ZLECENIE}:
	\begin{enumerate}
		\item \msg{REQ\_ZLECENIE} Dodaj proces do \var{QUEUE\_ZLECENIE}.
		\item \msg{ACK\_ZLECENIE} Zignoruj
		\item \msg{KONIEC\_BIURO}, \msg{START\_SZKIELET}, \msg{KONIEC\_SZKIELET} Dostarcz wiadomości, w kolejności odebrania, po przejściu do stanu \state{CZEKAJ}. Może się zdarzyć, jeśli jedna z wiadomości \msg{OFFER} lub \msg{ZLECENIE} jeszcze nie doszła do procesu.
		\item  Pozostałe jak w stanie \state{START}.
	\end{enumerate}
	
	\item Proces po przejściu do stanu \state{MAM\_ZLECENIE} wysyła wiadomość \msg{OFFER} z id zlecenia do wszystkich procesów o innej specjalizacji.
	
	\item Po otrzymaniu dwóch wiadomości \msg{OFFER} z id równym id otrzymanego zlecenia oraz jeśli $z_p >= id$ zlecenia proces przechodzi do stanu \state{CZEKAJ}.
\end{enumerate}

\subsection{Dostęp do biur}
\begin{enumerate}
	\item Zamiast przejścia do stanu \state{CZEKAJ} proces-ogon wysyła do wszystkich innych procesów-ogonów komunikat \msg{REQ\_BIURO} i przechodzi do stanu \state{CZEKAJ\_BIURO}.
	
	Reakcje na wiadomości w stanie \state{CZEKAJ\_BIURO}:
	\begin{enumerate}
		\item \msg{REQ\_BIURO} Jeśli wysłana prośba ma priorytet mniejszy, wyślij \msg{ACK\_BIURO}, w przeciwnym przypadku dodaj proces do \var{QUEUE\_BIURO}
		\item \msg{ACK\_BIURO} Jeśli jest to odpowiedź na ostatnią wysłaną wiadomość \msg{REQ\_BIURO} zwiększ licznik ACK o 1, w p.p. zignoruj
		\item  Pozostałe jak w stanie \state{START}
	\end{enumerate}
	
	\item Po otrzymaniu co najmniej $n_o - b$ odpowiedzi \msg{ACK\_BIURO} proces przechodzi do stanu \state{PRACA\_BIURO}.
	
	Reakcje na wiadomości w stanie \state{PRACA\_BIURO}:
	\begin{enumerate}
		\item \msg{REQ\_BIURO} Dodaj proces do \var{QUEUE\_BIURO}.
		\item \msg{ACK\_BIURO} Zignoruj
		\item  Pozostałe jak w stanie \state{START}.
	\end{enumerate}
	
	\item Po pewnym czasie proces kończy pracę i wysyła \msg{ACK\_BIURO} do wszystkich procesów w \var{QUEUE\_BIURO}. Następnie wysyła \msg{KONIEC\_BIURO} do procesów w swoim zespole i przechodzi do stanu \state{CZEKAJ}.
	
	\item Reakcje na wiadomości w stanie \state{CZEKAJ}:
	\begin{enumerate}
		\item \msg{KONIEC\_BIURO} Jeśli odbiorcą jest proces-tułów zacznij staranie się o dostęp do szkieletu. W przeciwnym przypadku zignoruj.
		\item \msg{START\_SZKIELET} Przejdź do stanu \msg{PRACA\_SZKIELET}.
		\item Pozostałe jak w stanie \state{START}.
	\end{enumerate}
\end{enumerate}

\subsection{Dostęp do szkieletów}
\begin{enumerate}
	\item Proces-tułów po otrzymaniu komunikatu \msg{KONIEC\_BIURO} wysyła do wszystkich innych procesów-tułowiów komunikat \msg{REQ\_SZKIELET} i przechodzi do stanu \state{CZEKAJ\_SZKIELET}.
	
	Reakcje na wiadomości w stanie \state{CZEKAJ\_SZKIELET}:
	\begin{enumerate}
		\item \msg{REQ\_SZKIELET} Jeśli wysłana prośba ma priorytet mniejszy, wyślij \msg{ACK\_SZKIELET}, w przeciwnym przypadku dodaj proces do \var{QUEUE\_SZKIELET}
		\item \msg{ACK\_SZKIELET} Jeśli jest to odpowiedź na ostatnią wysłaną wiadomość \msg{REQ\_SZKIELET} zwiększ licznik ACK o 1, w p.p. zignoruj
		\item  Pozostałe jak w stanie \state{START}
	\end{enumerate}
	
	\item Jeśli otrzyma co najmniej $n_t - s$ odpowiedzi \msg{ACK\_SZKIELET} wysyła do procesów w swoim zespole wiadomość \msg{START\_SZKIELET} i przechodzi do stanu \state{PRACA\_SZKIELET}.
	
	Reakcje na wiadomości w stanie \state{PRACA\_SZKIELET}:
	\begin{enumerate}
		\item \msg{REQ\_SZKIELET} Dodaj proces do \var{QUEUE\_SZKIELET}.
		\item \msg{ACK\_SZKIELET} Zignoruj
		\item \msg{KONIEC\_SZKIELET} Zwiększ licznik zakończonych części o 1
		\item  Pozostałe jak w stanie \state{START}.
	\end{enumerate}
	
	\item Po pewnym czasie proces kończy pracę i wysyła wiadomość \msg{KONIEC\_SZKIELET} do innych procesów w swojej grupie.
	
	\item Jeżeli proces otrzymał dwie wiadomości \msg{KONIEC\_SZKIELET} i sam ją wysłał to przechodzi do stanu \state{START}.
	
	\item Proces-tułów przed przejściem do stanu \state{START} wysyła wiadomości \msg{ACK\_SZKIELET} do procesów w kolejce \msg{QUEUE\_SZKIELET}.
\end{enumerate}

\section{Inne pomysły i optymalizacje}
\begin{enumerate}
	\item Po wyjściu ze stanu \state{PRACA\_SZKIELET} tylko proces-tułów może czekać na wiadomości \msg{KONIEC\_SZKIELET} - wtedy również nie musi ich wysyłać.
	\item Przy dostępie do zleceń można dodać tablicę \var{LAST\_RECEIVED} z czasem otrzymania ostatniej wiadomości. Proces czeka wtedy, aż $n_x - 1$ wartości będzie większych lub równych niż czas wysłania \msg{REQ\_ZLECENIE}. Gdyby wszystkie procesy zaczęły się ubiegać, jeden z nich otrzyma \msg{REQ\_ZLECENIE} o niższych priorytetach i nie musi czekać na \msg{ACK\_ZLECENIE}. Proces wiedząc, że wysłał \msg{REQ\_ZLECENIE} o niższym priorytecie nie musi wtedy wysyłać \msg{ACK\_ZLECENIE} - tutaj można też dodać tablicę \var{LAST\_SEND}.
\end{enumerate}

\subsection{Złożoność Komunikacyjna}
Przez złożoność komunikacyjną w tym zadaniu rozumiana jest liczba komunikatów potrzebna do wykonania uprzednio wydanego zlecenia. Kluczowym spostrzeżeniem w obliczaniu złożoności komunikacyjnej jest zauważenie, że każdy komunikat można przypisać do jakiegoś zlecenia.:
\begin{enumerate}
	\item[\msg{ZLECENIE}] - Przypisana do właśnie wysyłanego zlecenia
	\item[\msg{ACK\_...}] - Przypisana do zlecenia, które jest przypisane do requesta, na które to ACK odpowiada
	\item[\msg{REQ\_ZLECENIE}] - Przypisana do zlecenia wykonywanego przez specjalistę po odebraniu odpowiedniej ilości ACKów.
	\item[\msg{OFFER}] Przypisana do zlecenia, które zostało przyjęte.
\end{enumerate}
Pozostałe komunikaty są przypisywane do obecnie wykonywanego zlecenia przez specjalistów (związane ze szkieletem i biurem - start i koniec pracy, requesty). Dzięki takiemu przypisaniu na gruncie teoretycznym (na poziomie algorytmu te przypisanie jest niewidoczne) możliwa jest analiza tego, ile komunikatów powstaje w związku z pojedynczym zleceniem. Warto też zauważyć następującą równość: \(n=n_o+n_t+n_g+1\) - liczba wszystkich procesów to liczba procesów o 3 specjalizacjach plus generator zleceń.
\begin{enumerate}
	\item Dostęp do zleceń i dobór do grupy:
	\begin{description}
		\item[\(n-1\)] - wysyłanie zleceń przez generator - ZLECENIE
		\item[\(2(n_g-1+n_t-1+n_o-1)=2n-8\)] - przyjmowanie  i odbieranie REQ/ACK\_ZLECENIE
		\item[\((n_o+n_t)+(n_g+n_o)+(n_g+n_t)=2n-2\)] - wysyłanie OFFERA do procesów o różnej specjalizacji.
	\end{description} 
	Wszystkie procesy w sumie mają złożoność komunikacyjną \(5n-11\)
	
	
	\item Dostęp do biur:
	\begin{description}
		\item[\(2(n_o-1)\)] przyjmuje i odbiera REQ/ACK\_BIURO od innych procesów-ogonów
		\item[\(2\)] wysłanie komunikatu KONIEC\_BIURO do pozostałych specjalistów
	\end{description} 
	W sumie \(2n_o\)
	\item Dostęp do szkieletów:
	\begin{description}
		\item[\(2(n_t-1)\)] przyjmuje i odbiera REQ/ACK\_SZKIELET od innych procesów-szkieletów
		\item[\(2\)] wysłanie komunikatu START\_SZKIELET do pozostałych specjalistów przez proces-tułów
		\item[\(3*2\)] wysłanie komunikatu KONIEC\_SZKIELET do pozostałych specjalistów przez każdego specjalistę
	\end{description}
	W sumie \(2n_t+6\)
	\item W sumie komunikacyjna - liczba komunikatów związanych z pojedynczym zleceniem - wynosi \(5n+2n_o+2n_t-5\).
\end{enumerate}

\subsection{Złożoność Czasowa}
Złożoność czasowa dla pojedynczego zlecenia jest liczona jako liczba rund, w których wysyłane są komunikaty przy pomijalnie małym czasie lokalnego przetwarazania i jednostkowym czasie wysyłania komunikatu.
\begin{enumerate}
	\item Dostęp do zleceń i dobór do grupy:
	\begin{description}
		\item[1] - wysyłanie \msg{ZLECENIE} przez generator i \msg{REQ\_ZLECENIE} przez specjalistów
		\item[2] - odbieranie i wysyłanie \msg{ACK\_ZLECENIE} przez specjalistów
		\item[3] - wysłanie \msg{OFFER} przez specjalistów z nr zlecenia
	\end{description}
	W sumie 3.
	
	\item Dostęp do biur:
	\begin{description}
		\item[4] wysyłanie \msg{REQ\_BIURO}
		\item[5] wysyłanie \msg{KONIEC\_BIURO} do kompanów po zakończeniu przcy w biurze przez ogon
		\item[6] wysyłanie \msg{ACK\_BIURO} do oczekujących procesów
	\end{description}
	W sumie 3.
	
	\item Dostęp do szkieletów:
	\begin{description}
		\item[7] wysyłanie \msg{REQ\_SZKIELET} przez proces-tułów
		\item[8] wysyłanie \msg{START\_SZKIELET} przez proces-tułów do kompanów po zdobyciu odpowiedniej ilości ACKów
		\item[9] wysyłanie \msg{KONIEC\_SZKIELET} przez proces do kompanów po zakończeniu pracy ze szkieletem.
		\item[10] wysyłanie \msg{ACK\_SZKIELET} przez proces-tułów do pozostałych procesów-tułowiów
	\end{description}
	W sumie 4.
	\item W sumie złożoność czasowa dla pojedynczego zlecenia wynosi 10.
	\item Złożoność będzie się komplikowała przy analizie \(m\) zleceń przy \(k\) specjalistach poszczególnych typów: O ile procesy 4-10. muszą być wykonane po przyjęciu zlecenia, o tyle procesy 2-3. nie muszą (1. musi zostać wykonany - należy wysłać zlecenie, aby proces-ogon mógł przejść do biura). Przy \(k=2\) przez 10 rund wykonywane jest pierwsze zlecenie, od 2.-10. rundy wykonywane jest 2. zlecenie - zlecone w 2. rundzie, po wysłaniu \msg{REQ\_ZLECENIE} przez 2 proces i otrzymaniu odpowiedzi równocześnie dostając zlecenie - dalej natomiast zlecenia będą wysyłane, ale nikt ich nie będzie przyjmował - aż do 11. rundy, kiedy obydwie trójki procesów wyjdą ze szkieletu i znowu wejdą w cykl zlecenia, przyjmując 2 z pozostałych, ciągle generowanych zleceń. Złożoność będzie wynosić \(10\lceil \frac{m}{2} \rceil\). Analogicznie, dla 3 procesów o każdej specjalizacji złożoność wyniosłaby \(10\lceil\frac{m}{3} \rceil\). Później, procesy będą kończyć asynchronicznie, bo 4. trójka procesów nie pójdzie dalej po wysłanie \msg{OFFER}, w oczekiwaniu na \msg{ZLECENIE}. Kolejne procesy będą się kończyły w rundach: \(10-10-10-11-20-20-20-21-30 \dots\); analogicznie, dla \(k \in <4;9>\) proocesy kończą się w rundach \(10-10-10-11-...-10+k-3-20-20-20\). Dla \(k \in <10;\infty)\) zlecenia będą się kończyły w rundach \(10-10-10-11- \dots - 19?-20-21-22-23 \dots\). (Niekoniecznie 19, ale wtedy będzie zamiast 19/18 będą 20-tki) Reasumując:
	\begin{description}
		\item \(k \in \{2,3\} \Rightarrow o=10\lceil \frac{m}{k} \rceil\)
		\item \(k \in <4;9> \Rightarrow o=10\lceil \frac{m}{k} \rceil+
		\begin{cases}
			0 \iff ((m-1)\ \bmod k)-2<=0\\
			((m-1) \bmod k)-2\ \text{otherwise}\\
		\end{cases}\)
		\item \(k \in <10;\infty) \Rightarrow o=10+max(m, 3)-3\)
		
		Gdzie \(o\) to liczba rund - złożoność czasowa.
	\end{description}
\end{enumerate}

\subsection{Pojemność kanału}
Konieczna pojemność kanału komunikacyjnego może być potencjalnie nieskończona - po pierwsze, ponieważ proces może nie zareagować na nieskończoną liczbę zleceń od generatora zleceń. Po drugie, jeśli jakiś proces np. wskrzesza smoka przez dłuższy czas i nie odpowiada na komunikaty, pozostałe procesy mogą wysyłać mu \msg{REQ\_SZKIELET}, nie dostawać odpowiedzi i i tak wchodzić do sekcji krytycznej (bo s>1), wychodzić z niej i ponawiać proces w nieskończoność. 

\end{document}
