\documentclass[12pt]{article}
\usepackage[utf8]{inputenc}
\usepackage[T1]{fontenc}
\usepackage{lmodern}
\usepackage{amsmath}

\usepackage[svgnames]{xcolor}
\usepackage[a4paper,bindingoffset=0.2in,%
left=1in,right=1in,top=1in,bottom=1.5in,%
footskip=.25in]{geometry}
\pagenumbering{gobble}
\usepackage[colorlinks=true, linkcolor=Black, urlcolor=Blue]{hyperref}

\newcommand{\state}[1]{\texttt{\textbf{#1}}}
\newcommand{\msg}[1]{\texttt{\emph{#1}}}
\newcommand{\var}[1]{\texttt{#1}}

\begin{document}
\title{Algorytm - Wskrzeszanie smoków}
\author{Sebastian Michoń 136770, Marcin Zatorski 136834}
\date{}
\maketitle

\section{Wstęp}
Celem zadania było opracowanie programu rozwiązującego problem rozproszonej sekcji krytycznej. W zadaniu są dwa rodzaje procesów. Generator zleceń wysyła co jakiś czas zlecenia do innych procesów. Zlecenia są rozróżnialne i mają przypisane id (zaczynając od 1), zgodnie z kolejnością generacji. Pozostałe procesy to profesjonaliści o trzech możliwych specjalizacjach: głowa, tułów lub ogon. Do wykonania zlecenia potrzeba po jednym specjaliście z każdej specjalizacji. Przed wykonaniem zlecenia jeden ze specjalistów (ogon) musi wykonać pracę w biurze. Biur jest $b$, i są one nierozróżnialne. Po wykonaniu pracy w biurze można wykonać zlecenie - wskrzesić smoka. Potrzeba do tego jednego ze szkieletów. Szkieletów jest $s$ i są one nierozróżnialne. Szkielety nie znikają po wskrzeszeniu smoka. Za dostęp do szkieletów odpowiada proces-tułów.

Proces o specjalizacji x będzie nazywany ``proces-x'' (np. proces o specjalizacji głowa to proces-głowa). Liczby procesów o poszczególnych specjalizacjach będą oznaczane przez: procesy-głowy: $n_g$, procesy-tułowie: $n_t$, procesy-ogony: $n_o$.

\subsection{Założenia}
	\begin{enumerate}
		\item Biura o liczności $b$ i szkielety o liczności $s$ są zasobami nierozróżnialnymi.
		\item Procesy wiedzą jaką rolę pełni każdy inny proces.
		\item Proces o specjalizacji x będzie nazywany "proces-x" (np. proces o specjalizacji głowa to proces-głowa)
		\item Zlecenia są rozróżnialne i mają przypisane id (numerowane od 1), zgodnie z kolejnością generacji.
		\item Liczby procesów o poszczególnych specjalizacjach będą oznaczane przez: procesy-głowy: $n_g$, procesy-tułowie: $n_t$, procesy-ogony: $n_o$.
	\end{enumerate}

\section{Zmienne}
\begin{description}
	\item[$z_p$] Liczba otrzymanych wiadomości \msg{TASK}
	\item[$z_c$] Liczba otrzymanych wiadomości \msg{REQ\_TASK} o priorytecie większym
	\item[$cnt$] Liczba otrzymanych wiadomości \msg{REQ\_TASK} o priorytecie mniejszym
	
	\item[\var{OFFERS}] Zbiór procesów od których otrzymano wiadomość \msg{OFFER}, razem z ich id zlecenia
	
	\item[\var{QUEUE\_OFFICE}] Kolejka komunikatów \msg{REQ\_OFFICE}, na które jeszcze nie wysłano odpowiedzi
	\item[\var{QUEUE\_SKELETON}] Kolejka komunikatów \msg{REQ\_SKELETON}, na które jeszcze nie wysłano odpowiedzi
\end{description}

\section{Wiadomości}
\begin{description}
	\item[\msg{TASK}] Informacja o nowym zleceniu
	\item[\msg{REQ\_TASK}] Informacja o chęci dostępu do zlecenia
	\item[\msg{ACK\_TASK}] Zaakceptowanie dostępu do zlecenia
	\item[\msg{OFFER}] Oferta przyłączenia do grupy, zawiera id zlecenia
	\item[\msg{REQ\_OFFICE}] Prośba o dostęp do biura
	\item[\msg{ACK\_OFFICE}] Zaakceptowanie dostępu do biura
	\item[\msg{END\_OFFICE}] Informacja o końcu pracy w biurze
	\item[\msg{REQ\_SKELETON}] Prośba o dostęp do szkieletu
	\item[\msg{ACK\_SKELETON}] Zaakceptowanie dostępu do szkieletu
	\item[\msg{START\_SKELETON}] Rozpoczęcie pracy ze szkieletem
	\item[\msg{END\_SKELETON}] Koniec pracy ze szkieletem
\end{description}
Wiadomości \msg{REQ\_TASK}, \msg{REQ\_OFFICE}, \msg{REQ\_SKELETON} zawierają priorytet procesu - parę: zegar Lamporta, identyfikator procesu. Wiadomości \msg{ACK\_TASK}, \msg{ACK\_OFFICE}, \msg{ACK\_SKELETON} zawierają informację o wiadomości \msg{REQ}, na którą są odpowiedzią.

\section{Stany}
\begin{description}
	\item[\state{START}] Stan początkowy procesu, zanim zacznie ubieganie się o zlecenie.
	\item[\state{WAIT\_TASK}] Stan oczekiwania na sekcję krytyczną zleceń.
	\item[\state{HAS\_TASK}] Stan po otrzymaniu zlecenia, przed przydzieleniem do grupy.
	\item[\state{HAS\_TEAM}] Stan po znalezieniu pozostałych specjalistów koniecznych do wskrzeszenia smoka.
	
	\item[\state{IDLE}] Stan po dobraniu do grupy, czekanie na skończenie pracy w biurze i rozpoczęcie pracy ze szkieletem
	\item[\state{WAIT\_OFFICE}] Oczekiwanie na otrzymanie dostępu do biura
	\item[\state{WORK\_OFFICE}] Praca w biurze
	\item[\state{FINISH\_OFFICE}] Zakończenie pracy w biurze
	
	\item[\state{WAIT\_SKELETON}] Oczekiwanie na otrzymanie dostępu do szkieletu
	\item[\state{WORK\_SKELETON}] Praca ze szkieletem i wskrzeszanie smoka
	\item[\state{FINISH\_SKELETON}] Zakończenie pracy ze szkieletem
\end{description}

\section{Wskrzeszanie}

\subsection{Dostęp do zleceń i dobór do grupy}
\begin{enumerate}
	\item Generator co jakiś czas wysyła wiadomość \msg{TASK} do wszystkich procesów poza samym sobą.
	
	\item Pozostałe procesy zaczynają w stanie \state{START}, natychmiast przechodzą z niego do \state{WAIT\_TASK}.
	
	Reakcje na wiadomości w stanie \state{START}:
	\begin{enumerate}
		\item \msg{REQ\_SKELETON} Odpowiedz komunikatem \msg{ACK\_SKELETON}
		\item \msg{REQ\_OFFICE} Odpowiedz komunikatem \msg{ACK\_OFFICE}
		\item \msg{REQ\_TASK} Odpowiedz komunikatem \msg{ACK\_TASK}. Zwiększ wartość $z_c$ o 1.
		\item \msg{OFFER} Dodaj id procesu do zbioru ofert \var{OFFERS} razem z id zlecenia, którego dotyczy oferta.
		\item \msg{TASK} Zwiększ wartość $z_p$ o 1
		\item \msg{ACK\_OFFICE}, \msg{ACK\_SKELETON}, \msg{ACK\_TASK} Zignoruj
	\end{enumerate}
		
	\item Proces w stanie \state{WAIT\_TASK} wysyła wiadomość \msg{REQ\_TASK} do wszystkich procesów o tej samej specjalizacji.
	
	Reakcje na wiadomości w stanie \state{WAIT\_TASK}:
	\begin{enumerate}
		\item \msg{REQ\_TASK} Wyślij \msg{ACK\_TASK}. Jeśli odebrana prośba ma priorytet mniejszy to zwiększ $cnt$ o 1. W przeciwnym wypadku zwiększ $z_c$ o 1.
		\item \msg{ACK\_TASK} Jeśli jest to odpowiedź na ostatnią wysłaną wiadomość \msg{REQ\_TASK} zwiększ licznik ACK o 1, w p.p. zignoruj.
		\item Pozostałe jak w stanie \state{START}.
	\end{enumerate}
		
	\item Po otrzymaniu $n_x - 1$ odpowiedzi \msg{ACK\_TASK} proces przechodzi do stanu \state{HAS\_TASK}. Id zlecenia jest równe $z_c + 1$ (to zlecenie może jeszcze nie być wygenerowane). Następnie proces aktualizuje zmienne: $z_c$ += $cnt + 1$, $cnt = 0$.
	\begin{description}
		\item[$n_x$] liczba procesów o tej samej specjalizacji
		\item[$z_c$] liczba otrzymanych wiadomości \msg{REQ\_TASK} o priorytecie większym.
	\end{description}
			
	\item Reakcje na wiadomości w stanie \state{HAS\_TASK} i \state{HAS\_TEAM}:
	\begin{enumerate}
		\item \msg{REQ\_TASK} Wysyłane jest \msg{ACK\_TASK}, \(z_c\) jest zwiększane o 1.
		\item \msg{ACK\_TASK} Zignoruj
		\item \msg{END\_OFFICE}, \msg{START\_SKELETON}, \msg{END\_SKELETON} Dostarcz wiadomości, w kolejności odebrania, po przejściu do stanu \state{IDLE}. Może się zdarzyć, jeśli jedna z wiadomości \msg{OFFER} lub \msg{TASK} jeszcze nie doszła do procesu.
		\item  Pozostałe jak w stanie \state{START}.
	\end{enumerate}
	
	\item Proces po przejściu do stanu \state{HAS\_TASK} wysyła wiadomość \msg{OFFER} z id zlecenia do wszystkich procesów o innej specjalizacji.
	
	\item Po otrzymaniu dwóch wiadomości \msg{OFFER} z id równym id otrzymanego zlecenia proces przechodzi do stanu \state{HAS\_TEAM}.
	
	\item Jeśli w stanie \state{HAS\_TEAM} $z_p >= id$ proces przechodzi do stanu \state{IDLE}.
\end{enumerate}

\subsection{Dostęp do biur}
\begin{enumerate}
	\item Zamiast przejścia do stanu \state{IDLE} proces-ogon wysyła do wszystkich innych procesów-ogonów komunikat \msg{REQ\_OFFICE} i przechodzi do stanu \state{WAIT\_OFFICE}.
	
	Reakcje na wiadomości w stanie \state{WAIT\_OFFICE}:
	\begin{enumerate}
		\item \msg{REQ\_OFFICE} Jeśli wysłana prośba ma priorytet mniejszy, wyślij \msg{ACK\_OFFICE}, w przeciwnym przypadku dodaj proces do \var{QUEUE\_OFFICE}
		\item \msg{ACK\_OFFICE} Jeśli jest to odpowiedź na ostatnią wysłaną wiadomość \msg{REQ\_OFFICE} zwiększ licznik ACK o 1, w p.p. zignoruj
		\item  Pozostałe jak w stanie \state{START}
	\end{enumerate}
	
	\item Po otrzymaniu co najmniej $n_o - b$ odpowiedzi \msg{ACK\_OFFICE} proces przechodzi do stanu \state{WORK\_OFFICE}.
	
	Reakcje na wiadomości w stanie \state{WORK\_OFFICE}:
	\begin{enumerate}
		\item \msg{REQ\_OFFICE} Dodaj proces do \var{QUEUE\_OFFICE}.
		\item \msg{ACK\_OFFICE} Zignoruj
		\item  Pozostałe jak w stanie \state{START}.
	\end{enumerate}
	
	\item Po pewnym czasie proces kończy pracę i przechodzi do stanu \state{FINISH\_OFFICE}. Następnie wysyła \msg{ACK\_OFFICE} do wszystkich procesów w \var{QUEUE\_OFFICE}. Wysyła również \msg{END\_OFFICE} do procesu-tułowia.
	
	\item Reakcje na wiadomości w stanie \state{IDLE} i \state{FINISH\_OFFICE}:
	\begin{enumerate}
		\item \msg{END\_OFFICE} Jeśli odbiorcą jest proces-tułów zacznij staranie się o dostęp do szkieletu. W przeciwnym przypadku zignoruj.
		\item \msg{START\_SKELETON} Przejdź do stanu \msg{WORK\_SKELETON}.
		\item Pozostałe jak w stanie \state{START}.
	\end{enumerate}
\end{enumerate}

\subsection{Dostęp do szkieletów}
\begin{enumerate}
	\item Proces-tułów po otrzymaniu komunikatu \msg{END\_OFFICE} wysyła do wszystkich innych procesów-tułowiów komunikat \msg{REQ\_SKELETON} i przechodzi do stanu \state{WAIT\_SKELETON}.
	
	Reakcje na wiadomości w stanie \state{WAIT\_SKELETON}:
	\begin{enumerate}
		\item \msg{REQ\_SKELETON} Jeśli wysłana prośba ma priorytet mniejszy, wyślij \msg{ACK\_SKELETON}, w przeciwnym przypadku dodaj proces do \var{QUEUE\_SKELETON}
		\item \msg{ACK\_SKELETON} Jeśli jest to odpowiedź na ostatnią wysłaną wiadomość \msg{REQ\_SKELETON} zwiększ licznik ACK o 1, w p.p. zignoruj
		\item  Pozostałe jak w stanie \state{START}
	\end{enumerate}
	
	\item Jeśli proces-tułów otrzyma co najmniej $n_t - s$ odpowiedzi \msg{ACK\_SKELETON} wysyła do procesów w swoim zespole wiadomość \msg{START\_SKELETON} i przechodzi do stanu \state{WORK\_SKELETON}.
	
	Reakcje na wiadomości w stanie \state{WORK\_SKELETON}:
	\begin{enumerate}
		\item \msg{REQ\_SKELETON} Dodaj proces do \var{QUEUE\_SKELETON}.
		\item \msg{ACK\_SKELETON} Zignoruj
		\item \msg{END\_SKELETON} Zwiększ licznik zakończonych części o 1
		\item  Pozostałe jak w stanie \state{START}.
	\end{enumerate}
	
	\item Po zakończeniu pracy proces-tułów przechodzi do stanu \state{FINISH\_SKELETON}. Pozostałe procesy w zespole wysyłają do niego wiadomość \msg{END\_SKELETON} i przechodzą do stanu \state{START}.
	
	\item Jeżeli proces-tułów otrzymał dwie wiadomości \msg{END\_SKELETON} to  przechodzi do stanu \state{START}.
 	Wysyła też wiadomości \msg{ACK\_SKELETON} do procesów w kolejce \msg{QUEUE\_SKELETON}.
\end{enumerate}

\section{Inne pomysły i optymalizacje}
\subsection{Złożoność Komunikacyjna}
Złożoność komunikacyjna została obliczona jako liczba komunikatów potrzebnych do wykonania jednego zlecenia. Każdy komunikat jest powiązany z jednym zleceniem, a zatem każdemu komunikatowi można przypisać dokładnie jedno zlecenie:
\begin{enumerate}
	\item[\msg{TASK}] - Przypisana do wygenerowanego zlecenia
	\item[\msg{ACK\_...}] - Przypisana do tego zlecenia, które jest przypisane do requesta, na które to ACK odpowiada
	\item[\msg{REQ\_TASK}] - Przypisana do zlecenia wykonywanego przez specjalistę po odebraniu odpowiedniej ilości komunikatów \msg{ACK\_TASK}.
	\item[\msg{OFFER}] Przypisana do zlecenia, które zostało przyjęte.
\end{enumerate}
Pozostałe komunikaty są powiązane z zleceniem wykonywanym przez dany zespół. Liczba komunikatów wysłanych w ramach zlecenia to:
\begin{enumerate}
	\item Dostęp do zleceń i dobór do grupy:
	\begin{description}
		\item[\(n-1\)] - wysyłanie zleceń przez generator - \msg{TASK}
		\item[\(2(n_g-1+n_t-1+n_o-1)=2n-8\)] - przyjmowanie  i odbieranie \msg{REQ/ACK\_TASK}
		\item[\((n_o+n_t)+(n_g+n_o)+(n_g+n_t)=2n-2\)] - wysyłanie \msg{OFFER} do procesów o różnej specjalizacji.
	\end{description} 
	Wszystkie procesy w sumie mają złożoność komunikacyjną \(5n-11\)
	
	
	\item Dostęp do biur:
	\begin{description}
		\item[\(2(n_o-1)\)] przyjmuje i odbiera \msg{REQ/ACK\_OFFICE} od innych procesów-ogonów
		\item[\(1\)] wysłanie komunikatu \msg{END\_OFFICE} do procesu-tułowia przez proces-ogon.
	\end{description} 
	W sumie \(2n_o-1\)
	\item Dostęp do szkieletów:
	\begin{description}
		\item[\(2(n_t-1)\)] przyjmuje i odbiera \msg{REQ/ACK\_SKELETON} od innych procesów-szkieletów
		\item[\(2\)] wysłanie komunikatu \msg{START\_SKELETON} do pozostałych specjalistów przez proces-tułów
		\item[\(2\)] wysłanie komunikatu \msg{END\_SKELETON} do procesu-tułowia przez proces-ogon i proces-głowa.
	\end{description}
	W sumie \(2n_t+2\)
	\item W sumie komunikacyjna - liczba komunikatów związanych z pojedynczym zleceniem - wynosi \(5n+2n_o+2n_t-10\).
\end{enumerate}

\subsection{Złożoność Czasowa}
Złożoność czasowa dla pojedynczego zlecenia jest liczona jako liczba rund, w których wysyłane są komunikaty przy pomijalnie małym czasie lokalnego przetwarazania i jednostkowym czasie transmisji komunikatu.
\begin{enumerate}
	\item Dostęp do zleceń i dobór do grupy:
	\begin{description}
		\item[1] - wysyłanie \msg{TASK} przez generator i \msg{REQ\_TASK} przez specjalistów
		\item[2] - odbieranie i wysyłanie \msg{ACK\_TASK} przez specjalistów
		\item[3] - wysłanie \msg{OFFER} przez specjalistów z nr zlecenia
	\end{description}
	W sumie 3.
	
	\item Dostęp do biur:
	\begin{description}
		\item[4] wysyłanie \msg{REQ\_OFFICE}
		\item[5] wysyłanie \msg{END\_OFFICE} do procesu-tułowia po zakończeniu pracy w biurze przez ogon
		\item[6] wysyłanie \msg{ACK\_OFFICE} do oczekujących procesów
	\end{description}
	W sumie 3.
	
	\item Dostęp do szkieletów:
	\begin{description}
		\item[7] wysyłanie \msg{REQ\_SKELETON} przez proces-tułów
		\item[8] wysyłanie \msg{START\_SKELETON} przez proces-tułów do kompanów po zdobyciu odpowiedniej ilości ACKów
		\item[9] wysyłanie \msg{END\_SKELETON} przez procesy do procesu-tułowia po zakończeniu pracy ze szkieletem.
		\item[10] wysyłanie \msg{ACK\_SKELETON} przez proces-tułów do pozostałych procesów-tułowiów
	\end{description}
	W sumie 4.
	\item W sumie złożoność czasowa dla pojedynczego zlecenia wynosi 10.
	
\end{enumerate}

\subsection{Pojemność kanału}
Konieczna pojemność kanału komunikacyjnego może być potencjalnie nieskończona - choćby dlatego, że proces może nie zareagować na nieskończoną liczbę zleceń od generatora zleceń.

\end{document}
